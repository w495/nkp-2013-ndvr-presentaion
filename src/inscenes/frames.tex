
\subsection{Кадры}

\frame{
    \frametitle{При совпадени относительных длин сцен}

    \begin{itemize}
        \item нельзя делать вывод об одинаковости сцен;
        \item сравнить начальные и конечные кадры:
        \begin{itemize}
            \item[${\color{zdarkred}\bullet}$] сравнивать попиксельно или на основе <<знакового представления>> с разными масштабами:
                \begin{itemize}
                    \item[${\color{zdarkgreen} +}$] просто;
                    \item[${\color{zdarkred} -}$] не устойчиво к трансформациям.
                \end{itemize}
%             \item сравнивать на основе <<знакового представления>>:
%                 \begin{itemize}
%                     \item[${\color{zdarkgreen} +}$]
%                         вычислительно просто, устойчиво к трансформациям яркости;
%                     \item[${\color{zdarkred} -}$]
%                         не устойчиво к трансформациям направления и сдвигам,
%                         требует полного перебора направлений.
%                 \end{itemize}
            \item[${\color{zdarkred}\bullet}$] искать особенности в каждой паре кадров, SIFT:
                % \begin{itemize}
                %     \item вычисляем градиент в каждом пикселе
                %     \item строим гистограммы направлений градиентов по прямоугольным областям
                %     \item вклад каждого пикселя взвешиваем по гауссиане с центром в центре окрестности
                %     \item сравниваем как вектора в $L_2$
                % \end{itemize}
                \begin{itemize}
                    \item[${\color{zdarkgreen} +}$] надежно, устойчиво к искажениям;
                    \item[${\color{zdarkred} -}$] долго, для каждой пары сцен придется перевычислять особенности.
                \end{itemize}
            \item[${\color{zdarkgreen}\bullet}$] вычислить GIST для нужных кадров проверяемой cцены;
            \item[${\color{zdarkgreen}\bullet}$] воспользоваться <<мешком слов>> и МОВ.
        \end{itemize}
    \end{itemize}
При сравнении удобно иметь набор уже сопоставленных сцен.
}

