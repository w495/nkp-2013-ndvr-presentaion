\subsection{Мешок слов}

\frame{
    \frametitle{Мешок слов}
    Предполагаем, что есть некоторый набор изображений, на которых можно обучиться.
    
    \vspace{10pt}

    \zbluebox{Обучение}{
        \scriptsize
        \begin{itemize}
            \item[${\color{zdarkblue}\vartriangleright}$]
                собираем множество фрагментов (на основе SIFT);
            \item[${\color{zdarkblue}\vartriangleright}$]
                кластеризуем и строим словарь;
            \item[${\color{zdarkblue}\vartriangleright}$]
                квантуем каждый фрагмент по словарю;
            \item[${\color{zdarkblue}\vartriangleright}$]
                считаем <<мешки слов>> для каждого изображения;
            \item[${\color{zdarkblue}\vartriangleright}$]
                обучаем МОВ на мешках слов.
        \end{itemize}
    }
    
    \vspace{10pt}

    \zgreenbox{Сопоставление}{
        \scriptsize
        \begin{itemize}
            \item[${\color{zdarkgreen}\vartriangleright}$]
                выбираем фрагменты из изображения (на основе SIFT);
            \item[${\color{zdarkgreen}\vartriangleright}$]
                квантуем каждый фрагмент по словарю;
            \item[${\color{zdarkgreen}\vartriangleright}$]
                строим <<мешок слов>> для изображения;
            \item[${\color{zdarkgreen}\vartriangleright}$]
                применяем классификатор МОВ.
        \end{itemize}
    }
}




