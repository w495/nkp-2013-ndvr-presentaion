\subsection{Дескриптор сцены}

\frame{
    \frametitle{Дескриптор сцены}



    \def\drawdescrpicture{
        \begin{center}
        \begin{scriptsize}
        \tikzstyle{rootf} = [
            draw=yellow!50!black!70,thick,
            minimum height=2cm,
            minimum width=2cm,
            top color=yellow!20,
            bottom color=yellow!60!black!20,
            decorate,decoration={random steps,segment length=3pt,amplitude=1pt}
        ]

        \tikzstyle{pointf} = [
            rectangle, rounded corners,
            thick,
            minimum size=1cm,
            draw=zdarkblue!50!black!50,
            top color=white,
            bottom color=zdarkblue!50!black!20
        ]

        \tikzstyle{shotf} = [
            rectangle, rounded corners,
            thick,
            minimum size=1.5cm,
            draw=red!100!black!50,
            top color=white,
            bottom color=red!50!black!20
        ]

        \tikzstyle{subshotf} = [
            rectangle, rounded corners,
            thick,
            minimum size=1cm,
            draw=red!70!black!50,
            top color=white,
            bottom color=red!30!black!20
        ]


        \tikzstyle{spacef} = [
            rectangle, rounded corners,
            thick,
            minimum size=1cm,
            draw=green!50!red!70!black!50,
            top color=white,
            bottom color=green!30!red!30!black!20
        ]

        \tikzstyle{framef} = [
            rectangle, rounded corners,
            thick,
            minimum size=1.5cm,
            draw=green!50!black!50,
            top color=white,
            bottom color=green!50!black!20
        ]

        \tikzstyle{bowf} = [
            rectangle, rounded corners,
            thick,
            minimum size=1cm,
            draw=green!70!red!10!black!50,
            top color=white,
            bottom color=green!70!red!10!black!20
        ]

        \tikzstyle{gistf} = [
            rectangle, rounded corners,
            thick,
            minimum size=1cm,
            draw=green!50!black!50,
            top color=white,
            bottom color=green!50!black!20
        ]


        \tikzstyle{googlef} = [
            rectangle, rounded corners,
            thick,
            minimum size=1cm,
            draw=green!70!blue!70!black!50,
            top color=white,
            bottom color=green!70!blue!70!black!20
        ]


        \tikzstyle{itemf} = [
            rectangle,thick,
            minimum width=2cm,
            minimum height=2cm,
            draw=teal!50!black!50,
            top color=white,
            bottom color=teal!50!black!20
        ]
        \begin{tikzpicture}[thick, node distance=2.5cm, text height=0.7ex, text depth=.25ex, auto]

            \node[pointf] (all) {{\normalsize Дескриптор}};
            \node[shotf, below left of=all] (shot) {\footnotesize
                \begin{tabular}{c}
                    отношения длины \\
                    сцены к~длинам \\
                    других сцен \\
                \end{tabular}
            };
                \node[subshotf, above left of=shot] (subshot) {\scriptsize
                    \begin{tabular}{c}
                        удобно хранить, \\
                        с~объединениями \\
                        соседних сцен \\
                    \end{tabular}
                };
                \node[spacef, below of=subshot] (space) {\tiny
                    \begin{tabular}{c} %% для длин \\
                        по $3$ предыдущим \\
                        \ ---~$6$~вариантов.
                    \end{tabular}
                };

            \node[framef, below right of=all] (frame) {\footnotesize
                \begin{tabular}{c}
                    Xарактеристики \\
                    {\scriptsize начального и конечного} \\
                    кадров \\
                \end{tabular}
            };

                \node[bowf,  below   of=frame] (bow) {\scriptsize
                    \begin{tabular}{c}
                        {\tt\color{red} или} \\
                        <<мешки слов>> \\
                    \end{tabular}
                };

                    \node[bowf,  below of=bow] (bowplus) {\tiny
                        \begin{tabular}{c}
                            лучше соответвует \\
                            предметной области\\
                        \end{tabular}
                    };

                \node[gistf, below right of=frame] (gist) {\scriptsize
                    \begin{tabular}{c}
                        {\tt\color{red} или} \\
                        GIST \\
                    \end{tabular}
                };
   

            \path[->, red!100!black!50, very thick] (all) edge (shot);
                \path[->, red!70!black!50, very thick] (shot) edge (subshot);
                \path[->, green!50!red!70!black!50, very thick] (subshot) edge (space);


            \path[->, green!50!black!50, very thick] (all) edge (frame);
                \path[->, green!70!red!10!black!50, very thick] (frame) edge (bow);
                \path[->, green!50!black!50, very thick] (frame) edge (gist);


        \end{tikzpicture}
        \end{scriptsize}
        \end{center}
    }

    %\drawdescrpicture

    \begin{enumerate}
        \item Вектор отношений длины сцены к длинам других сцен;
            \begin{itemize}
                \item удобно сразу хранить, с объединениями соседних сцен;
                \item для относительных длин
                    по трем предыдущим\ ---~$6$~вариантов.
            \end{itemize}
        \vspace{6pt}
        \item Xарактеристики начального и конечного кадров:
        \begin{itemize}
            \item или <<мешки слов>> начального и конечного кадров:
            \begin{itemize}
                \item[${\color{zdarkgreen}+}$] лучше соответвует предметной области,
                \item[${\color{zdarkred}-}$] потенциально бесконечный размер вектора гистограммы;
            \end{itemize}
            \item или GIST начального и конечного кадров;
            \begin{itemize}
                \item[${\color{zdarkgreen}+}$] не требует какого либо обучения,
                \item[${\color{zdarkred}-}$] менее точен.
            \end{itemize}
        \end{itemize}
    \end{enumerate}
%     \vspace{12pt}
%     \begin{itemize}
%         \item[${\color{red}\Rightarrow}$] объемный вектор, сравнивать не удобно.
%     \end{itemize}
}

% 
% | 1| 2|x3|
% | 1   |x3|
% | 2   |x3|
% | 1|   x3|
% | 2|   x3|
% |      x3|


% Rather than using random projections to define the bits in a code, several authors have
% pursued machine learning approaches. In [5] the authors used an autoencoder with several
% hidden layers. The architecture can be thought of as a restricted Boltzmann machine (RBM)
% in which there are only connections between layers and not within layers. In order to learn 32
% bits, the middle layer of the autoencoder has 32 hidden units, and noise was injected during
% training to encourage these bits to be as binary as possible. This method indeed gives codes
% that are much more compact than the E2LSH codes. In [9] they used multiple stacked RBMs
% to learn a non-linear mapping between input vector and code bits. Backpropagation using
% an Neighborhood Components Analysis (NCA) objective function was used to refine the
% weights in the network to preserve the neighborhood structure of the input space. Figure 1
% shows that the RBM gives much better performance compared to random bits. A simpler
% machine learning algorithm (Boosting SSC) was pursued in [10] who used adaBoost to
% classify a pair of input items as sishotar or nonsishotar. Each weak learner was a decision
% stump, and the output of all the weak learners on a given output is a binary code. Figure 1
% shows that this boosting procedure also works much better than E2LSH codes, although
% slightly worse than the RBMs1
% .


