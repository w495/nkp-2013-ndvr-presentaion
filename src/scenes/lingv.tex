
\subsection{Выравнивания}

\frame{
    \frametitle{Выравнивания длин}

    \vspace{20pt}
    \begin{tabular}{|r|l|l|l|l|l|l|l|}
        \hline  Время
            & 1 cек
            & 2 cек
            & 3 cек
            & 4 cек
            & 5 cек
            & 6 cек
            & 7 cек \\
        \hline
        \hline  Видео $v_1$
            & \multicolumn{2}{|l|}{ $\videoshot_{1, 1}$ }
            & \multicolumn{4}{|l|}{ $\videoshot_{1, 2}$ }
            &  $\videoshot_{1, 3}$\\
        \hline  Видео $v_2$
            & \multicolumn{6}{|l|}{ $\videoshot_{2, 1}$ }
            & $\videoshot_{2, 2}$ \\
        \hline
    \end{tabular}

    \vspace{12pt}
    
    Алгоритм Гейла-Черча для выравниваня длин предложений
    параллельных корпусов на разных языках
    \begin{itemize}
        \item требуется установить,
            что $v_1$ и $v_2$, <<переводы>> друг друга;
        \item когда лучше выравнивать, 
            {\it до} или {\it после} перехода к~относительным длинам:
            { \tiny
                \begin{itemize}
                    \item[до:] перевычислять относительные длины,
                    \item[после:] учитывать масштаб относительных длин;
                \end{itemize}
            }
        \item вычислительные затраты.
    \end{itemize}
    
%     Если длина текущей сцены одного видео меньше чем в 2 раза
%     длины текущей сцены другого видео, то текущую сцену первого видео
%     рассматривается вместе со следующей (алгоритм Гейла-Черча).

}
% 
% | . | . . . | |
% | . . . . . | |
% 
% 
% 1, 2, 0.5
% 1, 1.666
