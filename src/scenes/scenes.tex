
\subsection{Виды}

\begin{frame}[allowframebreaks]{Поиск на основе сцен}

    \orangebox{Кадр\ ---~{\it frame}, фотографический кадр}
    {\footnotesize
        \begin{itemize}
            \item[${\color{pacificorange} \Leftarrow}$]
                отдельная статическая картинка;
            \item[${\color{pacificorange} \Leftarrow}$]
                обозначим $\videoframe$.
        \end{itemize}
    }

    \vspace{12pt}
    \zgreenbox{Cъемка\ ---~{\it shot}, кинематографический кадр}
    {\footnotesize
        \begin{itemize}
            \item[${\color{zdarkgreen} \Leftarrow}$]
                множество фотографических кадров,
                единство процесса съемки;
            \item[${\color{zdarkgreen} \Leftarrow}$]
                обозначим $\videoshot$,
                $\videoframe \in \videoshot$;
            \item[${\color{zdarkgreen} \Leftarrow}$]
                часто называют <<сценой>>, 
                далее будем рассматривать, $\videoshot$, назвая сценой;
        \end{itemize}
    }
    \vspace{12pt}
    \zbluebox{Сцена\ ---~{\it scene}, монтажный кадр}
    {\footnotesize
        \begin{itemize}
            \item[${\color{zdarkblue} \Leftarrow}$]
                множество фотографических кадров,
                единство места и времени;
            \item[${\color{zdarkblue} \Leftarrow}$]
                обозначим $\videoscene$,
                $\videoframe \in \videoshot \subset \videoscene$.
        \end{itemize}
    }

    \zgreenbox{Сцена как <<съемка>>, кинематографический кадр}
    {
        ---~совокупность множества фотографических кадров $\videoframe$
        внутри временной области $\videoline$, кадры,
        которой $\videoframe_{\color{red} \videoshot, i}$
        значительно отличается от кадров соседних областей.
        \[
            \videoshot =
                \{
                    \videoframe_{\color{red} \videoshot, i}
                        | \videoframediff(\videoframe_{\color{red} \videoshot, i},
                            \videoframe_{\color{red} \videoshot, j})
                                < {\color{red} \varepsilon},
                            \videoframe_{\color{red} \videoshot, i},
                            \videoframe_{\color{red} \videoshot, j}
                            \in \videoline
                \}
        \]\[
            \videoframediff \text{\footnotesize \ ---~функция разности кадров}.
        \]
    }

    Аналогично можно ввести определение <<звуковой сцены>>,
    предварительно разделив звуковой сигнал на отсчеты.
\end{frame}
