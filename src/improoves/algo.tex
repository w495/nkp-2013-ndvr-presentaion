\subsection{Алгоритм}



\frame{
    \frametitle{Алгоритм поиска нечетких дубликатов видео}
    {     
    \begin{itemize} \footnotesize
        \item $\nv$\ ---~новое видео;
        \item $\setsv = \{\sv[1], \sv[2], \dots, \sv[n]\}$\ ---~исходные видео:
            \begin{itemize}
                \item[${\color{zdarkblue}\leftarrow}$]
                    {\scriptsize  $\setsv$ может быть пустым};
                \item[${\color{zdarkblue}\leftarrow}$]
                    {\scriptsize для непустого $\setsv$
                        вычислены дескрипторы сцен элементов.}
            \end{itemize} 
        \item[1.] Сравниваем дескриптор каждой
            сцены $\videoshot_{\color{red}\nv,i}$ из $\nv$\
            с дескриптором каждой сцены
            $\videoshot_{\color{red}\sv[k],j}$ из $\sv[k]$ в $L_2$.
        \item[2.] Если дескрипторы совпали $\nv$ c дескрипторами $\sv[k]$.
            на~некотором временном промежутке ,
            то считаем эту часть $\nv$\ ---~дубликатом $\sv[k]$,
            \begin{itemize}
                \item[] {\scriptsize несовпавшие
                    части $\nv$\ помещаем в $\setsv$}.
            \end{itemize}
        \item[3.] Если дескрипторы не совпали, то считаем $\nv$\ уникальным и
            добавляем в $\setsv$.
    \end{itemize}
    \vspace{4pt}
    \begin{itemize}
        \item[${\color{zdarkred}(-)}$] Cравнивать дескрипторы явно\ ---~не эффективно,
        \begin{itemize}
            \item[] {\scriptsize можно применить семантическое хеширование}.
        \end{itemize}
    \end{itemize}
    }
}


\subsection{Семантическое хеширование}

\frame{
    \frametitle{Семантическое хеширование}
    \begin{itemize}
        \item Введем бинарные подписи.
            \vspace{10pt}
        \item Подписи для близких в $L_2$ сцен должны быть близки.
            \vspace{10pt}
        \item  Локально чувстивтельное хеширование:
            \begin{enumerate}
                \item  Cлучайная проекция данных на прямую.
                \item  Случайно выберем порог,
                    пометив проекции $0$ или $1$.
                \item  С увеличением числа бит подпись
                    приближает $L_2$-метрику в исходных дескрипторах.
            \end{enumerate}
            \vspace{10pt}
        \item  Обучаемое хеширование.
    \end{itemize}
}

\subsection{Обучаемое хеширование}

\frame{
    \frametitle{Обучаемое хеширование}


    \bluebox{Cтимулирование (boosting)}
    {\footnotesize
        \begin{itemize}
            \item[$\Leftarrow$] BoostSSC;
            \item[$\Leftarrow$] Расстояние между дескрипторами вычисляется,
                как расстояние Хемминга.
        \end{itemize}
    }
    \vspace{12pt}
    \orangebox{Ограниченная машина Больцмана}
    {\footnotesize
        \begin{itemize}
            \item[${\color{pacificorange}\Leftarrow}$] связь только между слоями;
            \item[${\color{pacificorange}\Leftarrow}$] внутри слоев связи нет;
            \item[${\color{pacificorange}\Leftarrow}$]
                мощность слоев понижается от размера входного
                вектора до размера требуемого кода.
        \end{itemize}

    }
%     
%     \begin{enumerate}
%         \item  Cтимулирование (boosting).
%         \begin{itemize}
%             \item  BoostSSC;
%             \item  Расстояние между дескрипторами вычисляется, 
%                 как расстояние Хемминга
%         \end{itemize}
%         \item  C использованием ограниченной машины Больцмана:
%         \begin{itemize}
%             \item  связь только между слоями;
%             \item  внутри слоев связи нет;
%             \item  мощность слоев понижается от размера входного
%                 вектора до размера требуемого кода;
%         \end{itemize}
%     \end{enumerate}
}

\section{Результаты}

\subsection{Результаты и перспективы}

\frame{
    \frametitle{Результаты и перспективы}
    {\color{zdarkgreen} Результаты}:
        { \footnotesize
            \begin{itemize}
                \item предложен подход:
                \begin{itemize}
                    \item[${\color{zdarkblue}\vartriangleright}$] {\scriptsize относительные длины},
                    \item[${\color{zdarkblue}\vartriangleright}$] {\scriptsize выравнивания},
                    \item[${\color{zdarkblue}\vartriangleright}$] {\scriptsize дескрипторы сцен};
                \end{itemize}
                \item проведены эксперименты.
            \end{itemize}
        }
        \vspace{12pt}
    {\color{zdarkgreen} Дальнейшее развитие}
        { \footnotesize
            \begin{itemize}
                \item изменить алгоритм поска перемены сцены,
                    \begin{itemize}
                        \item[$\vartriangleright$] {\scriptsize сейчас используется {\tt ffmpeg}};
                    \end{itemize}
                \item хеширование на основе машины Больцмана;
                \item эксперименты на реальном множестве исxодных данных.
            \end{itemize}
        }
}

% 
%  \scriptsize
%     \begin{codebox}
%         \Procname{Базовый-алгоритм($\TE$, $\TR$)}
%         \li     $\forall \ \WE \in \SE \in \TE $ \Do
%         \li         $\forall \WR \in \SR \in \TR $ \Do
%         \li             $ t(\WE|\WR) \gets u, u \in \mathbb{R}$;
%                     \End
%                 \End
%         \li     \Com{ Инициализируем таблицу $t(\WE|\WR)$ одинаковыми значениями.}
%         \li     \Пока {не сойдется} \Do
%         \li         $\forall \ \WE \in \SE \in \TE $ \Do \Com{ Инициализируем остальные таблицы.}
%         \li             $\forall \WR \in \SR \in \TR $ \Do
%         \li                 $ counts(\WE|\WR) \gets  0$;  $\quad total(\WR) \gets  0$;
%                         \End
%                     \End
%         \li         $\forall \ \SE, \SR \in \TE, \TR$ \Do \Com{ Вычисляем нормализациию. }
%         \li             $\forall \ \WE \in \SE$ \Do
%         \li                 $stotal(\WE) \gets  0$;
%         \li                 $\forall \ \WR \in \SR$ \Do
%         \li                     $stotal(\WE) \gets stotal(\WE) + t(\WE|\WR)$;
%                             \End
%                         \End
%         \li             $\forall \ \WE \in \SE$ \Do \Com{ Собираем подсчеты. }
%         \li                 $\forall \ \WR \in \SR$ \Do
%         \li                     $counts(\WE|\WR) \gets counts(\WE|\WR) + \dfrac{t(\WE|\WR)}{stotal(\WE)}$;
%         \li                     $total(\WR) \gets total(\WR) + \dfrac{t(\WE|\WR)}{stotal(\WE)}$;
%                             \End
%                         \End
%                     \End
%         \li         $\forall \ \WE \in  \TE $ \Do \Com{ Оцениваем вероятность.}
%         \li             $\forall \WR \in \TR $ \Do
%         \li                 $ t(\WE|\WR) \gets \dfrac{counts(\WE|\WR)}{total(\WR)}$;
%                         \End
%                     \End
%                 \End
%     \end{codebox}